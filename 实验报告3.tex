% !TeX program = xelatex

\documentclass[UTF8]{ctexart}
\usepackage{graphicx}

\title{实践报告}
\author{高鹏宇\\ 学号:2302007025}
\date{}

\begin{document}

\maketitle

\section*{一、练习内容与结果}

\subsection*{(一)命令行环境样例}
\begin{enumerate}
  \item 删除指定目录:
  \begin{verbatim}
  rm -r directory_to_delete
  \end{verbatim}
  
  \item 复制文件到指定目录:
  \begin{verbatim}
  cp source_file.txt destination_directory/
  \end{verbatim}
  
  \item 查看文件内容:
  \begin{verbatim}
  cat file.txt
  \end{verbatim}
  
  \item 查找包含特定文本的文件:
  \begin{verbatim}
  grep "search_text" *
  \end{verbatim}
  
  \item 压缩文件或目录:
  \begin{verbatim}
  tar -czvf archive_name.tar.gz directory_to_compress
  \end{verbatim}
  
  \item 解压文件:
  \begin{verbatim}
  tar -xzvf archive_name.tar.gz
  \end{verbatim}
\end{enumerate}

\subsection*{(二)Python入门基础样例}
\begin{enumerate}
  \item 定义函数:
  \begin{verbatim}
  def greet(name):
      return f"Hello, {name}!"
  \end{verbatim}
  结果:定义了一个名为 greet 的函数,它接受一个参数 name 并返回一个格式化的问候语。
  
  \item 调用函数:
  \begin{verbatim}
  print(greet("Alice"))
  \end{verbatim}
  结果:打印输出 Hello, Alice! 到控制台。
  
  \item 列表操作:
  \begin{verbatim}
  fruits = ["apple", "banana", "cherry"]
  fruits.append("orange")
  \end{verbatim}
  结果:在列表 fruits 中添加了一个元素 "orange",现在 fruits 列表的内容是 ["apple", "banana", "cherry", "orange"]。
  
  \item 字典操作:
  \begin{verbatim}
  person = {"name": "Alice", "age": 25}
  print(person["name"])
  \end{verbatim}
  结果:打印输出 Alice 到控制台,这是字典 person 中键 "name" 对应的值。
  
  \item 异常处理:
  \begin{verbatim}
  try:
      result = 10 / 0
  except ZeroDivisionError:
      print("Cannot divide by zero!")
  \end{verbatim}
  结果:尝试执行除以零的操作,这会引发一个 ZeroDivisionError 异常。程序捕获这个异常并打印输出 Cannot divide by zero! 到控制台。
  
  \item 文件读写:
  \begin{verbatim}
  with open("file.txt", "w") as f:
      f.write("Hello, World!")
  \end{verbatim}
  结果:创建(或覆盖)一个名为 file.txt 的文件,并向其中写入字符串 Hello, World!。如果文件不存在,它会被创建;如果文件已存在,其内容会被覆盖。
\end{enumerate}

\subsection*{(三)Python视觉应用样例}
\begin{enumerate}
  \item 使用OpenCV读取图像:
  \begin{verbatim}
  import cv2
  image = cv2.imread('image.jpg')
  \end{verbatim}
  结果:加载名为 image.jpg 的图像文件到变量 image 中。如果文件不存在或路径错误,image 将是 None。
  
  \item 显示图像:
  \begin{verbatim}
  cv2.imshow('Image', image)
  cv2.waitKey(0)
  cv2.destroyAllWindows()
  \end{verbatim}
  结果:在窗口中显示图像。cv2.waitKey(0) 使得窗口持续打开直到任意键被按下。之后,关闭所有OpenCV创建的窗口
  
  \item 转换为灰度图像:
  \begin{verbatim}
  gray_image = cv2.cvtColor(image, cv2.COLOR_BGR2GRAY)
  \end{verbatim}
    结果:将 image 转换为灰度图像且保存。

  \item 保存图像:
  \begin{verbatim}
  cv2.imwrite('gray_image.jpg', gray_image)
  \end{verbatim}
  结果:将灰度图像保存。

  \item 使用Pillow打开图像:
  \begin{verbatim}
  from PIL import Image
  img = Image.open('image.jpg')
  \end{verbatim}
  结果:使用Pillow库打开 image.jpg 文件并存储在变量 img 中。
  
  \item 调整图像大小:
  \begin{verbatim}
  img_resized = img.resize((100, 100))
  \end{verbatim}
  结果:将图像的大小调整为 100x100 像素,并存储在变量中。
  
  \item 绘制图形(Matplotlib):
  \begin{verbatim}
  import matplotlib.pyplot as plt
  plt.plot([1, 2, 3], [4, 5, 6])
  plt.show()
  \end{verbatim}
  结果:在图形界面中绘制一个简单的线图,x轴为 [1, 2, 3],y轴为 [4, 5, 6]。
  
  \item 显示图像(Pillow):
  \begin{verbatim}
  img.show()
  \end{verbatim}
  结果:使用默认的图像查看器显示图像 img。
  
  \item 图像滤镜(Pillow):
  \begin{verbatim}
  from PIL import ImageFilter
  img_filtered = img.filter(ImageFilter.BLUR)
  \end{verbatim}
  结果:对图像应用模糊滤镜,并将结果存储在变量中。
  
  \item 使用OpenCV进行边缘检测:
  \begin{verbatim}
  edges = cv2.Canny(image, 100, 200)
  cv2.imshow('Edges', edges)
  \end{verbatim}
  结果:使用Canny算法对图像 image 进行边缘检测,并将结果存储在 edges 变量中。然后显示边缘检测的结果。
\end{enumerate}

\section*{二、解题感悟}

命令行环境是一个强大的工具,它允许用户通过文本命令直接与操作系统交互。在使用过程中,我发现大多数命令行命令在不同的操作系统(如Linux、macOS和Windows的特定版本)中都是通用的,这使得跨平台工作变得更加容易。

在使用Python的过程中,我发现相较于C语言,Python的语法更简单直观,大量的库和框架,可以用于各种应用,是一种非常泛用的工具。除此之外,得益于像OpenCV、Pillow和Matplotlib这样的库,Python在视觉应用方面也非常强大。我们可以使用OpenCV和Pillow,可以轻松进行图像读取、编辑、转换和保存。

总结来说,命令行环境、Python都是强大的工具,它们提供了广泛的功能和灵活性,可以帮助我们高效地完成任务。我们尽力去学会熟练掌握它们。

\end{document}