% !TeX program = xelatex

\documentclass[UTF8]{ctexart}
\usepackage{hyperref}

\title{实践报告}
\author{高鹏宇\\ 学号:2302007025}
\date{}

\begin{document}

\maketitle

\section*{一、练习内容和结果}

\subsection*{(一)、Git部分:}

\begin{enumerate}
  \item \textbf{git config –global:配置全局Git环境}
  \begin{itemize}
    \item 命令:
    \begin{verbatim}
    git config --global user.name "你的用户名"
    git config --global user.email "你的邮箱"
    \end{verbatim}
    \item 说明:设置全局用户名和邮箱,以便每次提交时自动使用这些信息。
  \end{itemize}

  \item \textbf{git init:初始化本地仓库}
  \begin{itemize}
    \item 命令:
    \begin{verbatim}
    git init
    \end{verbatim}
    \item 说明:在当前目录创建一个新的Git仓库。
  \end{itemize}

  \item \textbf{git add:添加文件到暂存区}
  \begin{itemize}
    \item 命令:
    \begin{verbatim}
    git add <file>
    \end{verbatim}
    \item 说明:将指定文件或所有更改添加到暂存区。
  \end{itemize}

  \item \textbf{git commit -m:提交到本地仓库}
  \begin{itemize}
    \item 命令:
    \begin{verbatim}
    git commit -m "提交信息"
    \end{verbatim}
    \item 说明:将暂存区的更改提交到本地仓库,并附上提交信息。
  \end{itemize}

  \item \textbf{git log:查看历史提交日志}
  \begin{itemize}
    \item 命令:
    \begin{verbatim}
    git log
    \end{verbatim}
    \item 说明:显示本地仓库的所有提交记录。
  \end{itemize}

  \item \textbf{git branch:查看分支}
  \begin{itemize}
    \item 命令:
    \begin{verbatim}
    git branch
    \end{verbatim}
    \item 说明:列出本地仓库的所有分支。
  \end{itemize}

  \item \textbf{git clone:克隆远程仓库}
  \begin{itemize}
    \item 命令:
    \begin{verbatim}
    git clone <repository-url>
    \end{verbatim}
    \item 说明:将远程仓库克隆到本地。
  \end{itemize}

  \item \textbf{git status:查看当前仓库状态}
  \begin{itemize}
    \item 命令:
    \begin{verbatim}
    git status
    \end{verbatim}
    \item 说明:显示当前工作目录的状态,包括未跟踪的文件、已修改但未暂存的文件等。
  \end{itemize}

  \item \textbf{git log filename:查看单个文件的提交历史}
  \begin{itemize}
    \item 命令:
    \begin{verbatim}
    git log -- <file>
    \end{verbatim}
    \item 说明:显示指定文件的提交历史。
  \end{itemize}

  \item \textbf{git rm:删除文件}
  \begin{itemize}
    \item 命令:
    \begin{verbatim}
    git rm <file>
    \end{verbatim}
    \item 说明:从Git仓库中删除文件,同时将此更改添加到暂存区。
  \end{itemize}
\end{enumerate}

\subsection*{(二)、LaTeX部分:}

\begin{enumerate}
  \item \textbf{声明类:}
  \begin{itemize}
    \item 命令:
    \begin{verbatim}
    \documentclass[UTF8]{ctexart}
    \end{verbatim}
    \item 说明:设置文档为中文文章。
  \end{itemize}

  \item \textbf{文档类型:}
  \begin{itemize}
    \item 命令:
    \begin{verbatim}
    \documentclass{article}
    \end{verbatim}
    \item 说明:设置文档类型为article,还有report和book可供选择。
  \end{itemize}

  \item \textbf{换行:}
  \begin{itemize}
    \item 单行换行:
    \begin{verbatim}
    \\[optional offset]
    \end{verbatim}
    \item 说明:在一行的末尾使用,offset可选,用于调整行间距。
  \end{itemize}

  \item \textbf{正文:}
  \begin{itemize}
    \item 命令:
    \begin{verbatim}
    \begin{document}
    ...
    \end{document}
    \end{verbatim}
    \item 说明:正文内容位于这两个命令之间。
  \end{itemize}

  \item \textbf{注释:}
  \begin{itemize}
    \item 单行注释:
    \begin{verbatim}
    % 注释内容
    \end{verbatim}
    \item 多行注释:
    \begin{verbatim}
    \iffalse
    注释内容
    \fi
    \end{verbatim}
  \end{itemize}

  \item \textbf{换新页:}
  \begin{itemize}
    \item 命令:
    \begin{verbatim}
    \newpage
    \end{verbatim}
    \item 说明:在文档中创建新的一页。
  \end{itemize}

  \item \textbf{标题级别:}
  \begin{itemize}
    \item 一级标题:
    \begin{verbatim}
    \section{一级标题}
    \end{verbatim}
    \item 二级标题:
    \begin{verbatim}
    \subsection{二级标题}
    \end{verbatim}
    \item 三级标题:
    \begin{verbatim}
    \subsubsection{三级标题}
    \end{verbatim}
  \end{itemize}

  \item \textbf{段落:}
  \begin{itemize}
    \item 命令:
    \begin{verbatim}
    \par
    \end{verbatim}
    \item 说明:表示一段的结束。
  \end{itemize}

  \item \textbf{字体:}
  \begin{itemize}
    \item 命令:
    \begin{verbatim}
    {\fontname 内容}
    \end{verbatim}
    \item 说明:设置字体,如\texttt{\songti}表示宋体。
  \end{itemize}

  \item \textbf{大小:}
  \begin{itemize}
    \item 命令:
    \begin{verbatim}
    {\fontsize{size}{baselineskip} 内容}
    \end{verbatim}
    \item 说明:设置字体大小,size为字体大小,baselineskip为行距。
  \end{itemize}
\end{enumerate}

\section*{二、解题感悟}

Git作为一种版本控制系统,让代码和文档的版本管理变得简单而高效。在学习过程中,我发现掌握上述的Git命令,可以让我们轻松地调试代码,解决问题,大大提高了开发效率和代码质量。

LaTex是一门论文语言,在各个领域都有很多的应用。通过学习LaTex,我基本掌握了使用和创建LaTex文档的方法。我发现LaTeX具有很强的拓展性,通过使用不同的宏包,几乎可以满足所有排版需。除此之外,LaTeX文档的源代码是纯文本格式,这意味着它们可以在任何文本编辑器中打开和编辑,也便于版本控制和备份。

通过Git和LaTeX的学习与实践,我不仅提升了技术技能,也对软件开发和文档撰写有了更深入的理解。这些工具是我们未来日常工作和学习中不可或缺的一部分。

Github地址:\url{https://github.com/Gao-py/class_homework}

\end{document}