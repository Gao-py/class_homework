% !TeX program = xelatex

\documentclass[UTF8]{ctexart}
\usepackage{hyperref}

\title{实践报告}
\author{高鹏宇\\ 学号:2302007025}
\date{}

\begin{document}

\maketitle

\section*{一、练习内容与结果}

\subsection*{(一)shell脚本}

\begin{enumerate}
  \item \textbf{使用ls命令列出所有文件(包括隐藏文件),并以人类可读的格式、按最近访问顺序排序,并以彩色文本显示输出结果。}
  \begin{itemize}
    \item 命令:
    \begin{verbatim}
    ls -la --time-style=+%h %d %n --color=auto
    \end{verbatim}
    \item 说明:
    \begin{itemize}
      \item -la:显示所有文件和目录,包括隐藏文件(以 . 开头的文件)。
      \item --time-style=+\%h \%d \%n:按最近访问时间排序。
      \item --color=auto:以彩色文本显示输出结果。
    \end{itemize}
  \end{itemize}

  \item \textbf{编写两个 bash 函数 marco 和 polo}
  \begin{itemize}
    \item 命令:
    \begin{verbatim}
    #!/usr/bin/env bash

    # marco 函数:保存当前目录
    marco() {
        marco_dir=$(pwd)
        echo "Marco: Current directory saved as $marco_dir"
    }

    # polo 函数:无论当前目录是什么,都返回 marco 保存的目录
    polo() {
        echo "Polo: Returning to $marco_dir"
        cd "$marco_dir"
    }
    \end{verbatim}
    \item 说明:
    \begin{itemize}
      \item 将上述代码保存到 marco.sh 文件中。
      \item 在终端中运行 source marco.sh 来加载函数。
      \item 运行 marco 来保存当前目录。
      \item 无论当前目录是什么,运行 polo 都会返回到 marco 保存的目录。
    \end{itemize}
  \end{itemize}

\item \textbf{随机数脚本}
  \begin{itemize}
    \item 命令:
    \begin{verbatim}
    #!/usr/bin/env bash

    # 生成一个0到99之间的随机数
    n=$(( RANDOM % 100 ))

    # 检查随机数是否等于42
    if [[ n -eq 42 ]]; then
        # 如果等于42,则打印错误信息到标准输出
        echo "Something went wrong"
        # 打印错误信息到标准错误输出
        >&2 echo "The error was using magic numbers"
        # 退出脚本并返回错误码1
        exit 1
    fi
    \end{verbatim}
    \item 说明:
    \begin{itemize}
      \item 使用 RANDOM 变量生成一个0到99之间的随机数,并将其赋值给变量 n。
      \item 使用 if 语句检查变量 n 是否等于42。
      \item 如果 n 等于42,脚本会执行两个 echo 命令:
      \begin{itemize}
        \item 第一个 echo 将错误信息打印到标准输出(通常是终端)。
        \item 第二个 echo 将错误信息重定向到标准错误输出(通常是终端)。
      \end{itemize}
      \item 然后,脚本通过 exit 1 命令退出,并返回一个非零的错误码,表示发生了错误。
    \end{itemize}
  \end{itemize}

  \item \textbf{编写脚本,运行直到出错并记录输出}
  \begin{itemize}
    \item 命令:
    \begin{verbatim}
    #!/usr/bin/env bash

    # 定义要运行的命令
    command_to_run="your_command_here"

    # 初始化尝试次数
    attempts=0

    # 循环运行命令,直到出错
    while ! $command_to_run; do
        # 增加尝试次数
        ((attempts++))

        # 捕获标准输出和标准错误
        # 将标准输出重定向到文件
        $command_to_run > output.log 2>&1
    done

    # 检查是否出错
    if [ $? -ne 0 ]; then
        # 将错误信息重定向到错误日志文件
        cat output.log > error.log

        echo "The command failed after $attempts attempts." >> error.log

        echo "The output was:"

        cat error.log
    else
        echo "The command did not fail after $attempts attempts."
    fi
    \end{verbatim}
    \item 说明:
    \begin{itemize}
      \item 请将代码替换为你想要运行的实际命令。
      \item 使用while循环运行命令,直到命令出错(即命令返回非零退出状态)。
      \item 每次命令运行时,都会将标准输出和标准错误重定向到output.log文件。
      \item 一旦命令出错,将 output.log 文件的内容复制到 error.log 文件,并记录尝试次数。
      \item 最后,脚本会输出错误日志文件的内容。
    \end{itemize}
  \end{itemize}

  \item \textbf{编写命令或脚本递归查找最近使用的文件}
  \begin{itemize}
    \item 命令:
    \begin{verbatim}
    find . -type f -printf "%T+ %p\n" | sort -r
    \end{verbatim}
    \item 说明:
    \begin{itemize}
      \item find . -type f -printf "%T+ %p\n":递归查找所有文件,并打印它们的最后访问时间(%T+)和路径(%p)。
      \item sort -r:按时间倒序排序。
    \end{itemize}
  \end{itemize}

  \item \textbf{创建目录}
  \begin{itemize}
    \item 命令:
    \begin{verbatim}
    mkdir new_directory
    \end{verbatim}
    \item 说明:
    \begin{itemize}
      \item mkdir:创建新目录。
    \end{itemize}
  \end{itemize}

  \item \textbf{统计当前目录下文件数目}
  \begin{itemize}
    \item 命令:
    \begin{verbatim}
    find . -type f | wc -l
    \end{verbatim}
    \item 说明:
    \begin{itemize}
      \item find . -type f:查找当前目录下的所有文件。
      \item wc -l:计算行数,即文件数量。
    \end{itemize}
  \end{itemize}

  \item \textbf{复制文件}
  \begin{itemize}
    \item 命令:
    \begin{verbatim}
    cp /path/to/source.txt /path/to/destination.txt
    \end{verbatim}
    \item 说明:
    \begin{itemize}
      \item cp:复制文件。
      \item /path/to/source.txt:源文件路径。
      \item /path/to/destination.txt:目标文件路径。
    \end{itemize}
  \end{itemize}

  \item \textbf{递归查找并删除特定扩展名的文件}
  \begin{itemize}
    \item 命令:
    \begin{verbatim}
    find . -type f -name "*.log" -exec rm -f {} \;
    \end{verbatim}
    \item 说明:
    \begin{itemize}
      \item find . -type f -name "*.log":递归查找所有扩展名为 .log 的文件。
      \item -exec rm -f {} \;:对找到的每个文件执行 rm -f 命令进行删除。
    \end{itemize}
  \end{itemize}

  \item \textbf{将当前目录下的所有 .jpg 图片文件压缩为一个ZIP文件}
  \begin{itemize}
    \item 命令:
    \begin{verbatim}
    zip -r images.zip *.jpg
    \end{verbatim}
    \item 说明:
    \begin{itemize}
      \item zip -r images.zip:创建一个名为 images.zip 的压缩文件。
      \item *.jpg:将当前目录下所有 .jpg 文件添加到压缩文件中。
    \end{itemize}
  \end{itemize}
\end{enumerate}


\subsection*{(二)Vim使用}

\begin{enumerate}
  \item \textbf{打开文件}
  \begin{verbatim}
  vim filename
  \end{verbatim}

  \item \textbf{进入插入模式}
  \begin{itemize}
    \item i 进入插入模式,在光标前插入。
    \item I 进入插入模式,在行首插入。
    \item a 进入插入模式,在光标后插入。
    \item A 进入插入模式,在行尾插入。
  \end{itemize}

  \item \textbf{退出插入模式}
  \begin{verbatim}
  Esc 退出插入模式,返回普通模式。
  \end{verbatim}

  \item \textbf{保存文件}
  \begin{verbatim}
  :w 写入文件。
  :w filename 将文件保存为另一个名字。
  \end{verbatim}

  \item \textbf{退出Vim}
  \begin{verbatim}
  :q 退出Vim。
  :q! 强制退出,不保存更改。
  \end{verbatim}

  \item \textbf{保存并退出}
  \begin{verbatim}
  :wq 保存更改并退出。
  :x 同 :wq。
  \end{verbatim}

  \item \textbf{撤销和重做}
  \begin{verbatim}
  u 撤销。
  Ctrl + r 重做。
  \end{verbatim}

  \item \textbf{复制和粘贴}
  \begin{verbatim}
  yy 复制整行。
  3yy 复制三行。
  p 粘贴。
  d 删除(剪切)。
  dd 删除(剪切)整行。
  \end{verbatim}

  \item \textbf{查找和替换}
  \begin{verbatim}
  /pattern 查找“pattern”。
  n 跳转到下一个匹配项。
  N 跳转到上一个匹配项。
  :%s/old/new/g 全局替换“old”为“new”。
  \end{verbatim}

  \item \textbf{多文件编辑}
  \begin{verbatim}
  :e filename 打开另一个文件。
  :bn 切换到下一个文件。
  :bp 切换到上一个文件。
  \end{verbatim}

  \item \textbf{撤销历史}
  \begin{verbatim}
  :u 撤销。
  :undolist 列出撤销历史。
  \end{verbatim}

  \item \textbf{全局命令}
  \begin{verbatim}
  :g 执行全局命令,例如 :g/^$/d 删除所有空行。
  \end{verbatim}
\end{enumerate}

\section*{二、解题感悟}

在使用过程中,我发现熟练使用 Shell 脚本可以极大地提高自动化任务的效率。我们编写好合适的Shell 脚本,可以通过命令行执行复杂的任务,这对于批量处理文件和自动化工作流程非常有用。而Vim 作为一个强大的文本编辑器,通过熟练掌握其快捷键和命令,可以非常快速地编辑文本。除此之外,Vim 在文本处理方面非常强大,特别是对于大文件的编辑,它的性能和效率往往超过其他文本编辑器。无论是 Shell 还是 Vim,我们都需要持续学习和实践来提高个人能力。

Github地址:\url{https://github.com/Gao-py/class_homework}
\end{document}